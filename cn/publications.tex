\section{已发表论文}
% AGU style: https://publications.agu.org/agu-grammar-and-style-guide/
\newcommand{\CS}{*} % 通讯作者
\newcommand{\CF}{\textsuperscript{\#}} % 共同一作

\CS 通讯作者,\CF 共同一作
\begin{etaremune}
\item
    \Yao, Wu, S., Li, T., Bai, Y., Xiao, X., Hubbard, J., Wang, Y., Thant, M., \& Tong, P. (2022).
    Imaging the upper 10 km crustal shear-wave velocity structure of central Myanmar via a joint inversion of P-wave polarizations and receiver functions.
    \textit{Seismological Research Letters}.
	\DOI{10.1785/0220210292}.
\item
    Chen, J., Chen, G., Wu, H., \Yao, \& Tong, P. (2022).
    Adjoint tomography of NE Japan revealed by common-source double-difference traveltime data.
    \textit{Seismological Research Letters}.
    \DOI{10.1785/0220210317}.
\item
    Li, T., \Yao, Wu, S., Xu, M., \& Tong, P. (2022).
    Moho complexity in southern California revealed by local PmP and teleseismic Ps waves.
    \textit{Journal of Geophysical Research: Solid Earth}, \textit{127}, e2021JB023033.
    \DOI{10.1029/2021JB023033}.
\item
    Wu, S., \Yao, Wei, S., Hubbard, J., Wang, Y., Yin, M., Myo, T., Wang, X., Wang, K., Liu, T., Liu, Q., \& Tong, P. (2021).
    New insights into the structural heterogeneity and geodynamics of the Indo-Burma subduction zone from ambient noise tomography.
    \textit{Earth and Planetary Science Letters}, \textit{562}, 116856.
    \DOI{10.1016/j.epsl.2021.116856}.
\item
    \Yao, Liu, S., Wei, S., Hubbard, J., Huang, B., Chen, M., \& Tong, P. (2021).
    Slab models beneath Central Myanmar revealed by a Joint Inversion of regional and teleseismic traveltime data.
    \textit{Journal of Geophysical Research: Solid Earth}, \textit{126}, e2020JB020164.
    \DOI{10.1029/2020JB020164}.
\item
    Tong, P., \Yao, Liu, Q., Li, T., Wang, K., Liu, S., Cheng, Y., \& Wu, S. (2021).
    Crustal rotation and fluids: Factors for the 2019 Ridgecrest earthquake sequence?.
    \textit{Geophysical Research Letters}, \textit{48}, e2020GL090853.
    \DOI{10.1029/2020GL090853}.
\item
    Lythgoe, K., Inggrid, M., \& \Yao (2020).
    On waveform correlation measurement uncertainty with implications for temporal changes in inner core seismic waves.
    \textit{Physics of the Earth and Planetary Interiors}, \textit{309}, 106606.
    \DOI{10.1016/j.pepi.2020.106606}.
\item
    \Yao\CS, Tian, D., Sun, L., \& Wen, L. (2020).
    Comment on “Origin of temporal changes of inner-core seismic waves” by Yang and Song (2020).
    \textit{Earth and Planetary Science Letters}, \textit{553}, 116640.
    \DOI{10.1016/j.epsl.2020.116640}.
\item
    \Yao\CS, Tian, D., Sun, L., \& Wen, L. (2019).
    Temporal change of seismic Earth's inner core phases: inner core differential rotation or temporal change of inner core surface?.
    \textit{Journal of Geophysical Research: Solid Earth}, \textit{124}, 6720--6736.
    \DOI{10.1029/2019JB017532}.
\item
    \Yao\CS, Tian, D., Lu, Z., Sun, L., \& Wen, L. (2018).
    Triggered seismicity after North Korea's 3 September 2017 nuclear test.
    \textit{Seismological Research Letters}, \textit{89}(6), 2085--2093.
    \DOI{10.1785/0220180135}.
\item
    \Yao\CS, Tian, D., Sun, L., \& Wen, L. (2018).
	Source characteristics of North Korea's 3 September 2017 nuclear test.
    \textit{Seismological Research Letters}, \textit{89}(6), 2078--2084.
    \DOI{10.1785/0220180134}.
\item
    Tian, D., \Yao\CF, \& Wen, L. (2018).
    Collapse and earthquake swarm after North Korea's 3 September 2017 nuclear test.
    \textit{Geophysical Research Letters}, \textit{45}(9), 3976--3983.
    \DOI{10.1029/2018GL077649}.
\item
    温联星, 田冬冬, 姚家园. (2018).
    地球内核及其边界的结构特征和动力学过程.
    \textit{Chinese Journal of Geophysics}, \textit{61}(3), 803--818.
    \DOI{10.6038/cjg2018L0500}.
\item
    \Yao\CS, Sun, L., \& Wen, L. (2015).
    Two decades of temporal change of Earth's inner core boundary.
    \textit{Journal of Geophysical Research: Solid Earth}, \textit{120}(9), 6263--6283.
    \DOI{10.1002/2015JB012339}.
\item
    \Yao\CS, \& Wen, L. (2014).
    Seismic structure and ultra-low velocity zones at the base of the Earth's mantle beneath Southeast Asia.
    \textit{Physics of the Earth and Planetary Interiors}, \textit{233}, 103--111.
    \DOI{10.1016/j.pepi.2014.05.009}.
\end{etaremune}

%\subsection*{修改/审稿中}
%\begin{etaremune}
%\end{etaremune}

%\subsection*{准备中}
%\begin{etaremune}
%\end{etaremune}
