\section{学术报告}

\newcommand{\Invited}{\textbf{[邀请报告]}}

\begin{etaremune}
\item
    \Yao\
    缅甸区域的天然地震学成像研究.
    中国合肥市, 中国科学技术大学.
    2023 年 3 月 3日.
\item
    \Yao\
    体波走时层析成像揭示缅甸中部俯冲板片结构.
    中国南京市, 南京大学构造与地球物理青年学术论坛.
    2021 年 1 月 7 日.
\item
    \Yao\
    体波走时层析成像揭示缅甸中部俯冲板片形态.
    中国上海市, 同济大学第五届青年学者国际学术论坛.
    2020 年 5 月 30 日.
\item
    \Yao\
    缅甸区域地震和远震走时层析成像.
    美国加州旧金山, 2019 AGU 秋季会议.
    2019 年 12 月 9 日.
\item
    \Yao\
    朝鲜 2017 年核试验及其触发地震活动的地震学监测.
    中国武汉市, 中国科学院测量与地球物理研究所.
    2019 年 5 月 30 日.
\item
    \Yao\
    朝鲜 2017 年核试验及其触发地震活动的地震学监测.
    中国武汉市, 华中科技大学引力中心.
    2019 年 5 月 30 日.
\item
    \Yao\
    朝鲜 2017 年核试验及其触发地震的定位和震源特征.
    新加坡, 地球观测研究所.
    2018 年 9 月 14 日.
\item
    \Yao\
    朝鲜 2017 年核试验的地震学监测.
    中国北京市, 中国科学院地质与地球物理研究所.
    2018 年 6 月 15 日.
\item
    \Yao\
    地球内核震相的时间变化: 内核差速性旋转还是内核表面的时间变化?
    中国北京市, 中国科学院地质与地球物理研究所.
    2018 年 6 月 15 日.
\item
    \Yao\
    地球内核表面的时间变化和朝鲜 2017 年核试验的地震学监测.
    中国北京市, 中国地震局地震预测研究所.
    2018 年 6 月 14 日.
\item
    \Yao\
    朝鲜 2017 年核试验的地震学监测.
    中国北京市, 中国地震台网中心.
    2018 年 6 月 12 日.
\item
    \Yao\
    地球内核震相的时间变化: 内核差速性旋转还是内核表面的时间变化?
    中国北京市, 中国地球科学联合学术年会 2017.
    2017 年 10 月 17 日.
\item
    \Yao\
    地球内核表面的时间变化.
    中国北京市, 中国地震台网中心.
    2016 年 6 月 30 日.
\end{etaremune}
