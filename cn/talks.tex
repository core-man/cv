\section{学术报告}

\newcommand{\Invited}{\textbf{[邀请报告]}}

\begin{etaremune}
\item
    缅甸区域的天然地震学成像研究.
    中国合肥市, 中国科学技术大学.
    2023 年 3 月 3日.
\item
    体波走时层析成像揭示缅甸中部俯冲板片结构.
    中国南京市, 南京大学构造与地球物理青年学术论坛.
    2021 年 1 月 7 日.
\item
    体波走时层析成像揭示缅甸中部俯冲板片形态.
    中国上海市, 同济大学第五届青年学者国际学术论坛.
    2020 年 5 月 30 日.
\item
    Joint regional earthquake and teleseismic traveltime tomography of Myanmar.
    \textit{2019 AGU Fall Meeting, San Francisco}, CA, USA.
    Dec. 9, 2019.
\item
    朝鲜 2017 年核试验及其触发地震活动的地震学监测.
    中国武汉市, 中国科学院测量与地球物理研究所.
    2019 年 5 月 30 日.
\item
    朝鲜 2017 年核试验及其触发地震活动的地震学监测.
    中国武汉市, 华中科技大学引力中心.
    2019 年 5 月 30 日.
\item
    Location and source characteristics of North Korea's 2017 nuclear test and its triggered seismicity.
    \textit{Earth Observatory of Singapore}, Singapore.
    Sep. 14, 2018.
\item
    朝鲜 2017 年核试验的地震学监测.
    中国北京市, 中国科学院地质与地球物理研究所.
    2018 年 6 月 15 日.
\item
    地球内核震相的时间变化: 内核差速性旋转还是内核表面的时间变化?
    中国北京市, 中国科学院地质与地球物理研究所.
    2018 年 6 月 15 日.
\item
    地球内核表面的时间变化和朝鲜 2017 年核试验的地震学监测.
    中国北京市, 中国地震局地震预测研究所.
    2018 年 6 月 14 日.
\item
    朝鲜 2017 年核试验的地震学监测.
    中国北京市, 中国地震台网中心.
    2018 年 6 月 12 日.
\item
    地球内核震相的时间变化: 内核差速性旋转还是内核表面的时间变化?
    中国北京市, 中国地球科学联合学术年会 2017.
    2017 年 10 月 17 日.
\item
    地球内核表面的时间变化.
    中国北京市, 中国地震台网中心.
    2016 年 6 月 30 日.
\end{etaremune}
